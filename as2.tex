\documentclass[journal,12pt,twocolumn]{IEEEtran}
\usepackage[utf8]{inputenc}
 
\usepackage{tfrupee}
\usepackage{enumitem}
\usepackage{amsmath}
\usepackage{amssymb}


\begin{document}

\newcommand{\myvec}[1]{\ensuremath{\begin{pmatrix}#1\end{pmatrix}}}

\let\vec\mathbf


\title{Assignment 2}
\author{\textbf{Pathlavath Shankar (CS21BTECH11064)}}
\maketitle
\date {March 2022}


\textbf{\textit{Problem (iv), ICSE 12 2017:}}

Using L’Hospital’s Rule, evaluate:

 $\lim_{x\to\pi/2}$ ${xtanx- \pi/2secx}$

\textbf{\textit{Solution:}}

we know that,

if there is a function $f(x)=\frac{g(x)}{h(x)}$

then by L’Hospital’s Rule 
\begin{align}
\lim_{x\to x_0} f(x)&=\lim_{x\to x_0} \frac{g(x)}{h(x)}  \\
\label{eq:2}
&=\lim_{x\to x_0} \frac{g'(x)}{h'(x)}
\end{align}
so,by equation \eqref{eq:2},
\begin{align}
\lim_{x\to\pi/2} {xtanx- \pi/2secx}&=\lim_{x\to\pi/2}
 {\frac{\mathrm{d}(2xsinx-\pi)}{\mathrm{d}x}\frac{\mathrm{d}x}{\mathrm{d}(2cosx)}} \\
&=\lim_{x\to\pi/2}\frac{2xcosx + 2sinx}{-2sinx}
\end{align}
Now,putting value of x=$\pi/2$,we get  {-1}       \\

\end{document}